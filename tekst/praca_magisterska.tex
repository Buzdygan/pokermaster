\documentclass[licencjacka]{pracamgr}

\usepackage{polski}
\usepackage[utf8]{inputenc}
\usepackage{graphicx}
%\usepackage{float}
\usepackage{wrapfig}
\usepackage{hyperref}
\usepackage{placeins}
\usepackage{amsfonts}
\usepackage{amssymb}
\usepackage{amsmath}
\usepackage{setspace}

\singlespacing

\newtheorem{theorem}{Twierdzenie}[chapter]
\newtheorem{lemma}[theorem]{Lemat}
\newtheorem{proposition}[theorem]{Stwierdzenie}
\newtheorem{corollary}[theorem]{Wniosek}
\newtheorem{definition}[theorem]{Definicja}

\newenvironment{proof}[1][Dowód]{\begin{trivlist}
\item[\hskip \labelsep {\bfseries #1}]}{\end{trivlist}}
\newenvironment{example}[1][Przykład]{\begin{trivlist}
\item[\hskip \labelsep {\bfseries #1}]}{\end{trivlist}}
\newenvironment{remark}[1][Uwaga]{\begin{trivlist}
\item[\hskip \labelsep {\bfseries #1}]}{\end{trivlist}}

\newcommand{\qed}{\nobreak \ifvmode \relax \else
        \ifdim\lastskip<1.5em \hskip-\lastskip
            \hskip1.5em plus0em minus0.5em \fi \nobreak
                  \vrule height0.75em width0.5em depth0.25em\fi}

\allowdisplaybreaks


%\floatstyle{boxed}
%\restylefloat{figure}
%\floatplacement{figure}{H}

\author{Jakub Tlałka}
\nralbumu{292665}

\title{Heurystyczna modyfikacja techniki CFR w Pokerze}
\tytulang{Heuristic modification of CFR technique in Poker}

\kierunek{Informatyka}

\opiekun{dra Jakuba Pawlewicza\\
  Wydzia{\l} Matematyki Informatyki i Mechaniki\\
  }
  
\date{Sierpież 2014}

%TODO poprawić numerek
\dziedzina{
11.3 Informatyka\\
}

%TODO zmienić klasyfikacjż
\klasyfikacja{D. Maths\\
D.0. General\\}

%TODO uzupełnić sżowa kluczowe
\keywords{}
 
\newtheorem{defi}{Definicja}[section]

\begin{document}
\maketitle

\begin{abstract}
\end{abstract}

\tableofcontents

\chapter{Wstęp - sztuczna inteligencja w grach}

\chapter{Rozwój programów grających w Pokera}

\chapter{Metoda CFR}

Metoda CFR (Counterfactual Regret Minimization) to algorytm, który pozwala znajdować równowagę Nasha
w grach w postaci ekstensywnej. Jest to obecnie najlepsze rozwiązanie w dziedzinie badań nad
sztuczną inteligencją w Pokerze. Choć używane są głównie różne modyfikacje algorytmu, chciałbym
przedstawić tu jego bazową wersję.

\chapter{Modyfikacja}

\chapter{Implementacja}

\chapter{Symulacje na dwóch wariantach Pokera}

\chapter{Podsumowanie}


%TODO zmieniż bibliografiż
\begin{thebibliography}{99}
\addcontentsline{toc}{chapter}{Bibliografia}
\bibitem[1]{1} Guihua Gong and Guoliang Yu. $\;\;\;$ "Volume growth and positive scalar curvature. " $\;\;\;$. GAFA, Geom. funct. anal.
Vol. 10 (2000) 821 - 828 1016-443X/00/040821-8
\bibitem[2]{2} Piotr Nowak and Guoliang Yu. $\;\;\;$ "Large Scale Geometry. " $\;\;\;$ EMS Publishing House, Nashville/Warsaw (2012).
\bibitem[3]{3} Jean-Louis Tu $\;\;\;$ "Remarks on Yu's property A for discrete metric spaces and groups" $\;\;\;$ Bull. Soc. Math. France 129 (2001), no. 1, 115-139. MR 1871980 (2002j:58038)
\bibitem[4]{4} Mladen Bestvina. $\;\;\;$ "Groups as Metric Spaces." $\;\;\;$ University of Utah (2007)
\end{thebibliography}

\end{document}
