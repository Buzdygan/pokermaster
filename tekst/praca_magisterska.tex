\documentclass[licencjacka]{pracamgr}

\usepackage{polski}
\usepackage[utf8]{inputenc}
\usepackage{graphicx}
%\usepackage{float}
\usepackage{wrapfig}
\usepackage{hyperref}
\usepackage{placeins}
\usepackage{amsfonts}
\usepackage{amssymb}
\usepackage{amsmath}
\usepackage{setspace}

\singlespacing

\newtheorem{theorem}{Twierdzenie}[chapter]
\newtheorem{lemma}[theorem]{Lemat}
\newtheorem{proposition}[theorem]{Stwierdzenie}
\newtheorem{corollary}[theorem]{Wniosek}
\newtheorem{definition}[theorem]{Definicja}

\newenvironment{proof}[1][Dowód]{\begin{trivlist}
\item[\hskip \labelsep {\bfseries #1}]}{\end{trivlist}}
\newenvironment{example}[1][Przykład]{\begin{trivlist}
\item[\hskip \labelsep {\bfseries #1}]}{\end{trivlist}}
\newenvironment{remark}[1][Uwaga]{\begin{trivlist}
\item[\hskip \labelsep {\bfseries #1}]}{\end{trivlist}}

\newcommand{\qed}{\nobreak \ifvmode \relax \else
        \ifdim\lastskip<1.5em \hskip-\lastskip
            \hskip1.5em plus0em minus0.5em \fi \nobreak
                  \vrule height0.75em width0.5em depth0.25em\fi}

\allowdisplaybreaks


%\floatstyle{boxed}
%\restylefloat{figure}
%\floatplacement{figure}{H}

\author{Jakub Tlałka}
\nralbumu{292665}

\title{Heurystyczna modyfikacja techniki CFR w Pokerze}
\tytulang{Heuristic modification of CFR technique in Poker}

\kierunek{Informatyka}

\opiekun{dra Jakuba Pawlewicza\\
  Wydzia{\l} Matematyki Informatyki i Mechaniki\\
  }
  
\date{Sierpież 2014}

%TODO poprawić numerek
\dziedzina{
11.3 Informatyka\\
}

%TODO zmienić klasyfikacjż
\klasyfikacja{D. Maths\\
D.0. General\\}

%TODO uzupełnić sżowa kluczowe
\keywords{}
 
\newtheorem{defi}{Definicja}[section]

\begin{document}
\maketitle

\begin{abstract}
\end{abstract}

\tableofcontents

\chapter{Wstęp - sztuczna inteligencja w grach}

\chapter{Rozwój programów grających w Pokera}

\chapter{Metoda CFR}

Metoda CFR (Counterfactual Regret Minimization) to algorytm, który pozwala znajdować równowagę Nasha
w grach w postaci ekstensywnej. Jest to obecnie najlepsze rozwiązanie w dziedzinie badań nad
sztuczną inteligencją w Pokerze. Choć używane są głównie różne modyfikacje algorytmu, chciałbym
przedstawić tu jego bazową wersję.

\section{Terminologia}

\begin{itemize}

\item Zbiór graczy oznaczamy przez $N$.
\item Zbiór wszystkich możliwych ciągów akcji (historii) legalnych w grze nazywamy $H$. Ciągi, po których kończy
      się rozgrywka, należą do podzbioru $Z$. Oczywiście każdej historii jest przyporządkowany stan gry, który
      jest jej efektem.
\item Dla każdej historii $h$ zbiór akcji dostępnych w stanie gry jej odpowiadającym oznaczamy przez $A(h)$. Oczywiście
      każdy stan wyznacza jednoznacznie gracza wykonującego akcje.
\item Dla każdego gracza $i$, zbiór historii, w których ma on ruch, są pogrupowane w tzw zbiory informacyjne $I_i$. Dla zbioru
      informacyjnego $I_i$ oraz dowolnych dwóch historii $h_1, h_2 \in I_i$ zachodzi $A(h_1) = A(h_2) = A(I_i)$. Do jednego
      zbioru informacyjnego $I_i$ należą historie, które są parami tożsame po usunięciu z nich akcji niewidocznych dla gracza $i$
      (np w Pokerze są to akcje przydzielenia kart prywatnych graczom innym niż $i$).
\item Gracza losowego oznaczamy przez $c$. Funkcja $f_c(h, a)$ oznacza prawdopodobieństwo zaistnienia akcji $a$ dla historii $h$.
\item Strategię gracza $i$ oznaczamy przez $\sigma_i$. $\sigma_i(I_i, a)$ oznacza prawdopodobieństwo wykonania akcji $a$ przez
      gracza $i$ w zbiorze informacyjnym $I_i$ jeśli gra on strategią $\sigma_i$. Przyjmujemy $\sigma_c(h, a) = f_c(h, a)$.
      Przez $\sigma$ oznaczamy zespół strategii wszystkich graczy biorących udział w rozgrywce. Przez $\sigma_{-i}$ oznaczamy
      zespół strategii wszystkich graczy z wyjątkiem $i$.
\item Przez $\pi^{\sigma}(h)$ oznaczamy prawdopodobieństwo zajścia historii $h$ dla zespołu strategii $\sigma$.
      $\pi^{\sigma}(I)$ to suma tych prawdopodobieństw po $h \in I$. $\pi_i^{\sigma}(h)$ to prawdopodobieństwo że grając
      zgodnie ze strategią $\sigma$, gracz $i$ będzie podejmował akcje z odpowiadającej historii $h$. Analogicznie
      $\pi_{-i}^{\sigma}(h)$ oznacza prawdopodobieństwo że pozostali gracze będą podejmowali akcje z historii $h$.
\item Dla końcowych historii $h \in Z$ określamy funkcję użyteczności $u_i(h)$ oznaczającą zysk gracza $i$ w rozgrywce
      określonej przez $h$.

\end{itemize}

\noindent
W dalszej części będziemy zajmować się grami dwuosobowymi, przede wszystkim dwuosobową wersją Texas Holdem Poker.

\section{$\epsilon$-równowaga Nasha}

$\epsilon$-równowagą Nasha nazywamy zespół strategii $\sigma$, taki że:

\begin{align*}
u_1(\sigma) + \epsilon \geq  \max_{\sigma_1'} \, u_1(\sigma_1', \sigma_2) 
\end{align*}

\begin{align*}
u_2(\sigma) + \epsilon \geq  \max_{\sigma_2'} \, u_2(\sigma_1, \sigma_2') 
\end{align*}

\noindent
Metoda CFR znajduje $\epsilon$-równowagę Nasha dla $\epsilon$ malejącego proporcjonalnie do pierwiastka z liczby iteracji.

\section{Minimalizacja funkcji regretu}

W dalszej części przyjmujemy że rozgrywane są kolejne rundy indeksowane przez $t$. W każdej z rund
strategie graczy będą modyfikowane na podstawie dotychczasowej rozgrywki, w efekcie dając ciąg strategii
$(\sigma^t)$. Jako uśrednioną strategię gracza $i$ po $T$ rundach definiujemy:

\begin{align*}
\overline{\sigma}_i^T(I, a) = \frac{\sum\limits_{t=1}^T \pi_i^{\sigma^t}(I) \, \sigma^t(I, a)}{\sum\limits_{t=1}^T \, \pi_i^{\sigma^t}(I)}
\end{align*}

\noindent
Intuicyjnie funkcja regretu określa ile gracz mógłby zyskać zamieniając obecną strategię na najlepszą możliwą
przy założeniu że inni gracze pozostali by przy obecnych strategiach. Formalnie definiujemy średni całkowity regret gracza $i$
w rozgrywce o numerze $T$ przez:

\begin{align*}
R_i^T = \frac{1}{T} \, \max_{\sigma_i'} \sum\limits_{t=1}^T \, (u_i(\sigma_i', \sigma_{-i}^t) - u_i(\sigma^t))
\end{align*}

%todo dodaj referencje
\noindent
Znany wynik (referencja) mówi o tym, że w grze o sumie zerowej, jeśli po $T$ iteracjach średni całkowity regret obu graczy
jest mniejszy niż $\epsilon$ to uśredniony zespół strategii $\overline{\sigma}^T$ jest $2\epsilon$-równowagą Nasha.

\section{Funkcja regretu lokalnego}
Trudno jest minimalizować całkowity regret liczony:w




\chapter{Modyfikacja}

\chapter{Implementacja}

\chapter{Symulacje na dwóch wariantach Pokera}

\chapter{Podsumowanie}


\end{document}
